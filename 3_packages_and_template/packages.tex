%%%%%%%%%%%%%%%%%%%%%%%%%%%%%%%%%%%%%%%%%%%%%%%%%%%%%%%%%%%%%%%%%%%%%%%%%%
% STYLE
\usepackage[utf8]{inputenc} % most of not english characters!
\usepackage[T1]{fontenc} % accents and so on
\usepackage[catalan, spanish, english]{babel} % languages, maybe more than one
\usepackage{lmodern} % use latin modern fonts 
\usepackage{setspace} % linespacing -> makes it good
\usepackage{indentfirst} % indentation (space at the beginning) in new paragraphs
\usepackage{graphicx} % for importing pictures

%%%%%%%%%%%%%%%%%%%%%%%%%%%%%%%%%%%%%%%%%%%%%%%%%%%%%%%%%%%%%%%%%%%%%%%%%%
% SOME ARBITRARY NICE LATEX PACKAGES
\usepackage{hyperref} % turn all your internal references into hyperlinks
\usepackage[square, sort, numbers]{natbib} % Journal style bibliography managmenet. Bibliography files in the bibtex syntax are expected
\setcitestyle{numbers} % cite with numbers
\usepackage{verbatim} % inline verbatim/typewriter characters. Use \verb
\usepackage[pangram]{blindtext} % generates dummy text with \blindtext \Blindtext \blinddocument \Blinddocument
\usepackage[toc, acronym, style=super, automake]{glossaries} % to create a nice glossary
\usepackage{csquotes}

%%%%%%%%%%%%%%%%%%%%%%%%%%%%%%%%%%%%%%%%%%%%%%%%%%%%%%%%%%%%%%%%%%%%%%%%%%
% MATHS AND SCIENCE PACKAGES
\usepackage{amsmath, amsthm, amscd, amssymb} % the AMS packages:
% amsmath: provides miscellaneous enhancements for docs. containing mathematical formulas. For instance \align
% amsthm: theorems and so on
% amssymb: extended symbol collection
% amscd: commutative diagrams
\newtheorem{theorem}{Theorem} % so that it doesn't complain when adding theorems
\usepackage[all,cmtip]{xy} % more on diagrams, more arrows.
\usepackage{listings} % Write nice code in defined language. Use \begin{lstlisting}
\usepackage{physics} % brakets and other cool stuff